%% LyX 1.6.4.2 created this file.  For more info, see http://www.lyx.org/.
%% Do not edit unless you really know what you are doing.
%% https://github.com/yannickwurm/latexStyles

\documentclass[english,notitlepage]{report}
\usepackage{mathpazo}
\usepackage[T1]{fontenc}
\usepackage[utf8]{inputenc}
\usepackage[a4paper]{geometry}
\geometry{verbose,tmargin=3cm,bmargin=3cm,lmargin=2.5cm,rmargin=2.5cm}
\pagestyle{plain}
\setcounter{secnumdepth}{0}
\setcounter{tocdepth}{3}
\usepackage{xcolor}
\usepackage{babel}

\usepackage{float}
\usepackage{textcomp}
\usepackage{amstext}
\usepackage{graphicx}
\usepackage{setspace}
\usepackage[authoryear]{natbib}
\PassOptionsToPackage{normalem}{ulem}
\usepackage{ulem}
\doublespacing
\usepackage[unicode=true, 
 bookmarks=false,
 breaklinks=false,pdfborder={0 0 0},backref=false,colorlinks=false]
 {hyperref}
\hypersetup{
 pdfauthor={Jinliang Yang}}
 
\makeatletter

%%%%%%%%%%%%%%%%%%%%%%%%%%%%%% LyX specific LaTeX commands.
%% A simple dot to overcome graphicx limitations
\newcommand{\lyxdot}{.}

\providecolor{lyxadded}{rgb}{0,0,1}
\providecolor{lyxdeleted}{rgb}{1,0,0}
%% Change tracking with ulem
\newcommand{\lyxadded}[3]{{\texorpdfstring{\color{lyxadded}}{}#3}}
\newcommand{\lyxdeleted}[3]{{\texorpdfstring{\color{lyxdeleted}\sout{#3}}{}}}

%%%%%%%%%%%%%%%%%%%%%%%%%%%%%% User specified LaTeX commands.
\renewcommand{\harvardurl}{URL:\url}
\setcitestyle{citesep={;},aysep={}}
\usepackage{lineno}

\makeatother


\begin{document}
 \date{ } 
\title{The origin of biological nitrogen fixation in maize landraces around Totentepec}


\author{Jinliang Yang, David O’Donnell, Alan Bennett and Jeff Ross-Ibarra}

\maketitle
\vfill{}

\begin{description}
\begin{singlespace}
\item [{Author's~names~with~initials:}] \noindent Yannick Wurm, YW;
John Wang, JW; Laurent Keller, LK
\item [{Postal~address~for~all~authors:}] \noindent Department of Ecology
and Evolution, Biophore, University of Lausanne, 1015 Lausanne, Switzerland.
\item [{Keywords:}] \noindent \emph{Solenopsis invicta}, competition, reproduction,
timecourse, gene expression.
\item [{Corresponding~author:}] \noindent Yannick Wurm\\
Department of Ecology and Evolution, Biophore, \\
Université de Lausanne, 1015 Lausanne,\\
Switzerland\@.\\
Fax: +41 21 692 4165\\
yannick.wurm@unil.ch
\item [{Running~Title:}] \noindent Reproductive Opportunity and Gene Expression\end{singlespace}

\end{description}

\pagebreak{}
\begin{linenumbers}
\section{Abstract}

In species with social hierarchies, the death of dominant individuals
typically upheaves the social hierarchy and provides an opportunity
for subordinate individuals to become reproductives. Such a phenomenon
occurs in the monogyne form of the fire ant\emph{, Solenopsis invicta,
}where colonies typically contain a single wingless reproductive queen,
thousands of workers and hundreds of winged non-reproductive virgin
queens. Upon the death of the mother queen, many virgin queens shed
their wings and initiate reproductive development instead of departing
on a mating flight. Workers progressively execute almost all of them
over the following weeks. 

To identify the molecular changes that occur in virgin queens as they
perceive the loss of their mother queen and begin to compete for reproductive
dominance, we collected virgin queens before the loss of their mother
queen, six hours after orphaning and 24 hours after orphaning. Their
RNA was extracted and hybridized against microarrays to examine the
expression levels of approximately 10,000 genes. We identified 297
genes that were consistently differentially expressed after orphaning.
These include genes that are putatively involved in the signaling
and onset of reproductive development, as well as genes underlying
major physiological changes in the young queens. 

\pagebreak{}
\section{Introduction}

Maize \citep{Doyle1990}, with a global output exceeding 844 million tons in 2010, is the most vital crop in the world [3]. Aside from traditional use for human consumption, maize has important use as livestock feed, cooking oil, and biofuel [4]. The projected global human population of 9 billion in the year 2050 is expected to require 70\% more food than is presently produced, a major portion of which must come from maize [4]. With the advent of the Green Revolution, reported increases in yield have been due to intense application of synthetic nitrogen fertilizers. 60\% of such fertilizers are used for cereal production [5], which entirely doubled between 1966 and 2000 [4, 6]. Yet fertilizer-use by cereals is inefficient, at a rate of less than 50\% [7]. This inefficiency results in nitrate leaching into soil and groundwater supplies, thereby generating profound environmental concerns and health risks [8]. Furthermore, present rates of fertilizer over-application cannot be sustained [9], leaving an ever-growing need to develop alternative solutions for increased crop yield. This has led to recent efforts and accumulating evidence to indicate nitrogen fixation in cereal grains of agronomic importance [5, 8, 10, 11, 12]. As stated previously, our research collaborative has accrued evidence for biological nitrogen fixation in Totontepec maize (highlighted in Preliminary Results). With this foundation, I will work to genetically characterize nitrogen fixation and aerial root development in Zea mays.

In recent years, genetic mapping in maize has been greatly enhanced by development of the maize Nested Association Mapping panel [13]. With a joint-linkage association approach, this population has allowed for the genetic characterization of complex traits. Traits that were not intensely selected during maize domestication and improvement have trait architectures governed by numerous loci of small effect. Flowering time and leaf architecture, for instance, each are controlled by more than 30 QTL [14, 15]. This is in stark contrast to the genetic architectures for known domestication and improvement traits. Teosinte branched1 (tb1, governing apical dominance) and teosinte glume architecture1 (tga1, governing seed coat lignification) are two loci of extreme effect that govern radical differences between teosinte and modern maize domesticates [16, 17]. In addition, kernel carotenoid content, brought under intense selection within the past century due to its beneficial health effects for humans and livestock, is governed by only three loci of large effect [3, 18, 19, 20]. This forms the rationale for aim 1. Based upon the hypothesis that nitrogen fixation in maize is governed by aerial root development and mucilage production [to act as a chemo-attractant for diazotrophic microbes], this trait would have been present in ancestral teosinte populations – namely, Zea mays spp. Mexicana (hereafter Mexicana) which develops an abundance of aerial roots (Figure 4A). Following the model of trait architecture for pre-domestication traits, and including mild selection pressures by humans, I anticipate a moderate number of loci of small effect contributing to nitrogen fixation in maize.

Modern maize has a complex evolutionary history. Due to its tremendous phenotypic diversity [21, 22], maize originally was thought to be the product of multiple domestication events [23]. It now has been proven that all maize falls within a monophyletic clade derived from the wild teosinte Z. mays spp. Parviglumis (hereafter Parviglumis) [23], domesticated more than 8,700 years CBP in the Balsas River Valley [24]. Sequence analysis originally identified highland maize as genetically most similar to Parviglumis, which resides in lowland regions below ~1,800 m elevation [23]. Recent work has elucidated this paradox. Relative kinship between highland maize and Parviglumis has been affected by significant admixture between highland maize and the related Mexicana (Figure 4B) [25, 26, 27], which resides in highlands above ~1,700 m elevation [23] and diverged from Parviglumis ~61,000 years ago [28]. This forms the rationale for aims 2 and 3. Due to the presence of abundant aerial roots in Mexicana, and lack thereof within Parviglumis (Figure 4A), I hypothesize that aerial root formation, and thus nitrogen fixation in maize, were the products of adaptive introgression from Mexicana. I anticipate greater presence of nitrogen fixation and aerial root abundance in highland maize varieties, with selective sweeps indicative of selection for prominent aerial root development. Furthermore, I expect nitrogen fixation to be largely absent in improved lines, due to drastically reduced need for this process and selection for reallocation of energy to other physiological processes.


\section{Materials and methods}


\subsection{Ant collection and rearing}

Eight monogyne \emph{S. invicta }fire ant colonies, each containing
at least 50 winged virgin queens \textcolor{blue}{and between 5,000
and 10,000 workers}, were collected \textcolor{blue}{from a single
population} in Athens and Lexington, GA, USA in June 2006. \textcolor{blue}{There
is genetic variability between colonies, however it is moderate because
only few\emph{ }queens founded the North American \emph{S. invicta
}population when they were introduced from South America in the 1930s
\citep{Ross1993Effect-of-a-fou,RossShoemaker2008NumberOfFounders}.}
All colonies were returned to the laboratory and reared for one month
under standard conditions \citep{jouvenaz1977}. Queen- and male-destined
brood were \textcolor{blue}{identified by their large size and}
removed weekly \textcolor{blue}{to ensure that only field-reared
queens be used in for our experiment. Indeed, laboratory-reared queens
do not normally grow to the same sizes as field-reared queens and
rarely shed wings or initiate reproductive development after orphaning
(E. Vargo, NC State, Raleigh)}. We determined that each study colony
was of the monogyne social form using several lines of evidence. Nest
shape, nest density and worker size distribution were used to make
initial identifications of social form in the field \citep{Shoemaker:2006PopGenetics}.
Subsequently, monogyny was confirmed for each colony by the presence
of a single, highly physogastric, wingless queen. Finally, the social
form was further verified by electrophoretically detecting only the
\emph{B}\textbf{\emph{ }}but not the \emph{b }allele of \emph{Gp-9
}in pooled samples of 20 workers from each colony (lack of the \emph{b
}allele is diagnostic for monogyny in \emph{S. invicta} in the USA
\citep{ross:1997genetics,KellerGreenBeard1998,kriegerRoss02,Shoemaker:2006PopGenetics}). 


\subsection{Orphaning simulation, RNA isolation and microarray hybridization}

We removed the mother queen and collected virgin queens just before
orphaning as well as 6 and 24 hours after orphaning (subsequently
referred to as time points t$_{\text{0h}}$, t$_{\text{6h}}$ and
t$_{\text{24h}}$) to examine the \textcolor{blue}{onset of} the
molecular reaction to orphaning in virgin \emph{S. invicta }queens.
However, virgin queens emit pheromonal signals after orphaning that
are similar to those of an functional queen and can thus influence
each other \citep{Vargo1999PhysiolEntomol,FletcherCherixBlum1983}.
We attempted to minimize such effects and simplify interpretation
of results\textcolor{blue}{{} by taking advantage of the fact that }\textcolor{blue}{\emph{S.
invicta }}\textcolor{blue}{is a very opportunistic species that changes
nests often in its native habitat which includes large flood plains
\citep{Tschinkel:2006book}. We sampled queens according to the following
setup}: For t$_{\text{0h}}$, we haphazardly collected five virgin
queens from the foraging area of each source colony and individually
flash-froze them with liquid nitrogen in tubes containing 1g of 1.4mm
Zirconium Silicate beads (QuackenBush). We placed ten additional virgin
queens per source colony into individual \textcolor{blue}{small nests
}with 2g of mixed workers and brood\textcolor{blue}{. The density
of workers and brood was comparable to that found in the source colonies.}
\textcolor{blue}{All virgin queens isolated in this manner are expected
to shed their wings and initiate reproductive development \citep{Blum1981Pheromonal-Cont,Burns:2007fk}.}
\textcolor{blue}{We thus} simulated orphaning for a total of 80 virgin
queens from a total of eight source colonies.\textcolor{blue}{{} }We
harvested half of the virgin queens thus treated after 6 hours (t$_{\text{6h}}$)
and the remaining queens after 24 hours (t$_{\text{24h}}$). All collected
queens were individually flash-frozen immediately after collection
as described above. Samples were then stabilized until RNA isolation
by the addition of 900 $\mu$l of cold Trizol reagent (Invitrogen)
followed by homogenization with a FastPrep instrument (MP Biomedicals)
and storage at -80\textdegree{}C.\textcolor{blue}{{} In summary, we
had thus collected five queens at t$_{\text{0h}}$, five queens at
t$_{\text{6h}}$ and five queens at t$_{\text{24h}}$from each of
eight source colonies, constituting eight biological replicates for
our experiment (See also Supporting Figure 1). We chose to pool five
individuals from each replicate to reduce the impact of between-individual
differences \citep{kendziorski:2005}, and conducted eight replicates
to obtain sufficient statistical power with a feasible workload. In
comparison, other studies that examined the effects of social context
or mating used four replicates from a single }\textcolor{blue}{\emph{Drosophila
}}\textcolor{blue}{strain \citep{McGraw2004MatingSpermACPsFlies},
six replicates using different bees from a single colony \citep{grozingerMatingChanges},
six true biological replicates \citep{Renn2008Fish-and-chips,Roberge2008},
and examined pools of individuals from sixteen independent pairs of
ant colonies \citep{wang-plosgenetic}.}

Total RNA was isolated from all individuals using the Trizol protocol.
RNA was pooled from 5 individuals per source colony for each time
point and treated with DNA-free (Ambion). Subsequently, impurities
were filtered away with MicroCon-30 spin columns (Millipore), and
RNA quality was assessed on a 1\% agarose gel prior to amplification
using the MessageAmp II kit (Ambion). Amplified mRNA samples from
the eight colonies at three timepoints (t$_{\text{0h}}$, t$_{\text{6h}}$
and t$_{\text{24h}}$) were labeled, hybridized to microarrays made
from 22,560 independent fire ant cDNA spots \textcolor{blue}{\citep[Microarray construction described in ][]{wangESTs2007},}
and scanned as previously described \citep{wang-plosgenetic}. This
was done according to a dye-balanced loop design (Supporting Figure
1). For all procedures, precautions including randomization of sample
order were taken to avoid introducing unwanted biases. 


\subsection{Microarray analysis}

\textcolor{blue}{We followed a standard microarray analysis procedure,
guided by the documentation of the Bioconductor limma package \citep{smyth:2005,smyth2004limma,bioc2004}.
In brief, m}edian signal and background levels for each probe were
extracted from scanned microarray images using Axon Genepix software.
The limma\emph{ }2.16 package \citep{smyth2004limma} in R 2.8.1 \citep{r_reference:2007}
was used for normexp background correction, print-tip loess\emph{
}normalization within arrays, and aquantile normalization between
arrays \citep{Smyth2003Normalization-o}. The arrayQualityMetrics\emph{
}package \citep{arrayQualityMetricsKauffmann2009} and custom R scripts
were used for quality control. The 18,444 \emph{Solenopsis invicta}
cDNA spots yielding a single PCR band \citep{wangESTs2007} and passing
visual and automated inspection were used for analysis.\textcolor{blue}{{}
We constructed a design matrix incorporating effects for sampling
times (t$_{\text{0h}}$, t$_{\text{6h}}$ and t$_{\text{24h}}$),
biological replicate (eight colonies) and the two dyes. The model
that is fit to each gene may thus be represented as {}``$expression=timepoint+replicate+dye$''.
The limma}\textcolor{blue}{\emph{ }}\textcolor{blue}{package was used
for bayesian fitting of the model.} Differential expression was determined
for the contrasts {}``t$_{\text{24h}}$ vs. t$_{\text{0h}}$'',
{}``t$_{\text{24h}}$ vs. t$_{\text{6h}}$'', and {}``t$_{\text{6h}}$
vs. t$_{\text{0h}}$'' according to the nested F method in limma.
Briefly, a moderated F test determined that 521 microarray clones
were differentially expressed for at least one of the \textcolor{blue}{contrasts
with a 10\% False Discovery Rate \citep[FDR; ][]{BenjaminiHochberg1995}}.
Subsequently, significance of differential expression was assigned
to one or several contrasts. \textcolor{blue}{In comparison, the effects
of mating on honey bees queens were determined with 5\% FDR \citep{grozingerMatingChanges},
a comparison between fire ant workers from different social structures
used 10\% FDR \citep{wang-plosgenetic}, and the effect of the presence
of brood on honey bee workers was determined with 30\% FDR \citep{Alaux:2009qv}.}


\subsection{Sequence data, annotation and gene category analysis}

\textcolor{blue}{The published sequences of all microarray clones
\citep{wangESTs2007} were assembled along with data from two runs
of 454 sequencing of independently constructed cDNA libraries} (Y.
Wurm, D. Hahn and DD. Shoemaker; DH and DDS are at USDA-ARS, Gainesville).
High quality sequence information was \textcolor{blue}{obtained }for\emph{
}16,227 out of the 18,444 \emph{S. invicta }cDNA clones used for gene
expression analysis. This was also the case for 475 out of the 521
significantly differentially expressed clones.

Annotation was obtained via several methods. First, we ran NCBI BLASTX
2.2.16 to compare assembled fire ant sequences with the non-redundant
protein database (EMBL release 99). We retained informative gene descriptions
of hits with E-value < $10^{-5}$. Second, Gene Ontology (GO) \citep{GeneOntology200}
annotations were inferred using BLASTX as previously described \citep{Wurm:2009wo}.
Finally, each fire ant sequence was manually assigned a single descriptive
category. The manually assigned gene category putatively encapsulates
the general function of each sequence and is derived subjectively
by examining the SwissProt or Ensembl database entries of the five
best BLASTX hits (E-values < $10^{-5}$), with an emphasis on GO,
Interpro, and PANTHER annotations. \textcolor{blue}{The manual annotation
comprises a total of 34 general gene categories} (J. Wang, M. Nicolas
and L. Ometto, University of Lausanne, Lausanne).

Overrepresentation of manually assigned gene categories and GO categories
was determined, respectively, using exact one-sided Fisher tests in
R and the Elim test from the topGO Bioconductor package \citep{topGo2006}
limited to categories containing at least 10 fire ant genes. \textcolor{blue}{These
included 514 Biological Processes, 131 Cellular Components and 171
Molecular Functions.}


\subsection{Comparison with data from other species}

\textcolor{blue}{We wanted to determine the extent to which gene expression
differences linked to changes in social context and reproductive status
in this study are likely to play similar roles in other insects. To
do this, we downloaded lists of significantly over- and under-expressed
genes from studies that examined the transition to reproduction in
flies, bees and mosquitos \citep{LawniczakBegun2004FlyMating,McGraw2004MatingSpermACPsFlies,Rogers2008AnophelesMating,grozingerMatingChanges}
as well as the fixed differences in reproductive status between honey
bee queens and workers \citep{Grozinger:2007brainVirginQ.SterileAndReproductiveWorkers}.}
Mapping between microarray probes and coding sequences was either
provided by the study's authors \citep[for ][]{Grozinger:2007brainVirginQ.SterileAndReproductiveWorkers},
obtained by BLASTN of probe sequences to coding sequences \citep[for][]{grozingerMatingChanges}
or downloaded \citep[for][]{LawniczakBegun2004FlyMating,McGraw2004MatingSpermACPsFlies,Rogers2008AnophelesMating}
from BioMart \citep{Biomart2009}. \textcolor{blue}{Orthology information
was required to compare lists of signficant genes between ants and
the other species, however such information is practically nonexistant.
This is in part because and only partial transcriptome and no proteome
or genomic sequence data are published for ants.} To obtain orthology
information, we modified the Inparanoid ortholog identification pipeline
\citep{inparanoid} as follows: BLASTX\emph{ }was replaced with\emph{
}TBLASTX and stringency was reduced so that match areas must span
at least 25\% of the longer sequence with actual matching segments
aligning with at least 10\% of the longer sequence. We independently
ran this modified Inparanoid pipeline on the assembled fire ant sequences
and the complete set of coding sequences of each of the following
species: \emph{Drosophila melanogaster} (Flybase release 5.9), \emph{Apis
mellifera} (Honey Bee Official Gene Set pre-release 2) and \emph{Anopheles
gambiae} (AgamP3.4).

\textcolor{blue}{To determine the extent of overlap between two lists
of significant genes, we constructed a 2-by-2 contingency table containing:
the number of orthologous genes in both lists, the number of genes
examined in the relevant studies but not part of the significant lists,
and the numbers of genes that were examined in both studies but in
only one of the two lists of significant genes. Subsequently, we conducted
an exact one-sided Fisher test to determine whether the number of
genes in both lists was higher than would be expected by chance. Only
significant results (p<0.05) are reported}.

\textcolor{blue}{We determined the extents of overlap between two
lists of significant genes from our study (the 146 genes upregulated
at one point after orphaning and the 152 genes downregulated at one
point after orphaning) and the lists of genes from each of the other
studies. This was possible for a reduced set of genes that are both
putatively orthologous between fire ants and the species from the
other study and present on both the ant microarray and the microarray
used in the other study. }For the \citet{Grozinger:2007brainVirginQ.SterileAndReproductiveWorkers}
study, we \textcolor{blue}{report significant overlap comparing the
list of genes upregulated at one point after orphaning in our study}
with the list of \textcolor{blue}{549 bee genes in the Honey Bee Official
Gene Set} that were more highly expressed in \textcolor{blue}{honey
bee} queens than in reproductive workers \textcolor{blue}{as well
as the list of 619 bee genes that were more highly expressed in honey
bee queens than in sterile workers}. For the \citet{grozingerMatingChanges}
study, we \textcolor{blue}{report significant overlap comparing the
list of genes upregulated at one point after orphaning in our study}
with the list of 441 bee genes that were more highly expressed in
mated than virgin\textcolor{blue}{{} honey bee queens, as well as with
the list of 356 genes that were more highly expressed in honey bee
queens that were mated but not yet laying eggs than in queens that
were mated and egg-laying}. For all remaining comparisons of pairs
of lists of significant ant and honey bee genes, the overlaps were
either non-significant, or were not examined because they concerned
five genes or less.

For the \citet{Rogers2008AnophelesMating} study, we obtained results
comparing our \textcolor{blue}{two fire ant gene lists} with a combined
list of 1,663 \emph{Anopheles} genes that were either more highly
expressed in females 2h, 6h and 24h after mating than in virgins or
more highly expressed 6h than 2h or 24h than 6h after mating\textcolor{blue}{,
as well as with the complementary list of 1,586 genes}\textcolor{blue}{\emph{
}}\textcolor{blue}{that were less highly expressed in virgins than
in mated female }\textcolor{blue}{\emph{Anopheles}}. For the \citet{McGraw2004MatingSpermACPsFlies}
and \citet{LawniczakBegun2004FlyMating} studies, we compared our
results with all individual lists of \emph{Drosophila }genes that
were differentially expressed in response to different aspects of
mating, as well as with a combined list of all mating-response genes
they had identified. 


\section{Results}


\subsection{Differential gene expression after orphaning}

Four hundred seventy-five of the \textcolor{blue}{16,227 sequenced}
cDNA clones, putatively representing 297 genes, were significantly
differentially expressed between the samples of virgin queens collected
0 hours, 6 hours and 24 hours after orphaning (respectively t$_{\text{0h}}$,
t$_{\text{6h}}$ and t$_{\text{24h}}$). The remaining genes were
either expressed similarly before and after orphaning, were highly
variable between biological replicates, or yielded signals too weak
for reliable assessment of differential expression. Among the 297
significantly differentially expressed genes, four were upregulated
within 6 hours of orphaning, while one was downregulated. One hundred
forty-four genes were more highly expressed twenty-four hours after
orphaning than at t$\mathcal{_{\text{0h}}}$ or at t$\mathcal{_{\text{6h}}}$
including one of the four genes that was already upregulated after
6 hours, while a total of 152 genes were significantly downregulated
after 24 hours (Figure \ref{Fig:numbersOfGenes}). One of the genes
significantly upregulated after 6 hours was significantly downregulated
between 6 and 24 hours. The significant genes are listed in Supporting
Tables 1 and 2. These gene expression changes precede or are independent
of wing shedding since none of 40 virgin queens collected 6 hours
after orphaning and only three of 40 virgin queens collected 24 hours
after orphaning had shed their wings. 


\subsection{Gene set enrichment analysis}

We bioinformatically annotated the genes that were significantly upregulated
or downregulated after orphaning and compared their annotations with
the annotations of all genes examined on the microarray by using two
different annotation methods. From our \textcolor{blue}{manually
assigned} annotation categories,\emph{ }two gene categories\emph{
}were overrepresented among upregulated genes. These were \emph{proteasome
}(11 observed, 1.2 expected, exact one-sided Fisher test $p=1*10^{-7}$)
and \emph{protein transport} (10 observed, 1.5 expected, exact one-sided
Fisher test $p=4*10^{-6}$\emph{ }). No other \textcolor{blue}{manually
assigned annotation categories} were overrepresented among up or
downregulated genes. From the BLAST-inferred Gene Ontology categories,
several categories were overrepresented among up- and downregulated
genes (complete list in Table \ref{Table 1}). In particular, genes
putatively part of the \emph{proteasomal complex} were overrepresented
among the upregulated genes (7 observed, 0.7 expected, $p=0.0003$,\emph{
}topGO Elim\emph{ }test, adjusted for 10\% False Discovery Rate (FDR)).
Among downregulated genes, those putatively located in \emph{microsomes}
and involved in \emph{oxidation reduction }were overrepresented (respectively
6 observed, 0.5 expected, FDR adjusted\emph{ }topGO Elim test $p=0.0007$,
and 14 observed, 3.3 expected, FDR adjusted topGO Elim test$p=0.0005$).
Additionally, genes that putatively have \emph{aromatase activity}
were overrepresented among the significantly downregulated genes (5
observed, 0.3 expected, FDR adjusted\emph{ }topGO Elim test $p=0.0014$).
In fact, all five of these genes are putative \emph{Cytochrome P450}s.


\subsection{Genes related to Juvenile Hormone metabolism}

Among the 297 genes significantly differentially expressed in orphaned
compared to non-orphaned queens, five have sequence similarity to
genes from other species that are involved in Juvenile Hormone (JH)
metabolism or response. In particular, three putative JH esterases
were significantly downregulated after orphaning, while one was significantly
upregulated. Additionally, a putative JH epoxide hydrolase was significantly
downregulated after orphaning. Several putative JH inducible genes
as well as a putative JH esterase-binding gene showed non-significant
increases in expression level after orphaning (Figure \ref{fig:jhPlot}).


\subsection{Comparison of fire ant results with data from honey bees}

\textcolor{blue}{To determine whether the differentially expressed
genes identified in our study are also differently expressed between
reproductive and non-reproductive individuals in honey bees, we compared
our results with the studies of \citet{Grozinger:2007brainVirginQ.SterileAndReproductiveWorkers}
and \citet{grozingerMatingChanges}. The first of the two studies}
identified genes differentially expressed between brains of honey
bee queens and workers. We identified a subset of 902 ant-bee orthologs
examined in both that study and ours. Genes upregulated in orphaned
fire ant queens were enriched for genes upregulated in brains of queen
bees relative to brains of reproductive workers (12 observed, 7.5
expected, exact one-sided Fisher test $p=0.005$). \textcolor{blue}{There
was no significant overlap between other pairs of lists of genes from
the two studies.} Among the twelve genes that overlap between the
groups of significantly upregulated ant and bee genes (Supporting
Table 3), four are part of the manually assigned gene category \emph{proteasome}
(0.2 expected, exact one-sided Fisher test $p=1*10^{-4}$). 

The other study identified genes differentially expressed between
virgin and mated honey bee queens \citep{grozingerMatingChanges}.
Among 2,286 ant-bee orthologs examined in our study as well as the
bee study, 13 genes were more highly expressed in response to orphaning
in fire ants and in response to mating in honey bee queens (7.7 expected,
exact one-sided Fisher test $p=0.038$, genes listed in Supporting
Table 4). Among the thirteen genes that overlap between the two gene
lists, four are part of the manually assigned gene category \emph{proteasome
(}0.2 expected, exact one-sided Fisher test $p=1*10^{-4}$). \textcolor{blue}{There
was no significant overlap between other pairs of lists of genes from
the two studies.} 


\subsection{Comparison of fire ant results with data from dipterans}

\textcolor{blue}{To determine whether the differentially expressed
genes identified in our study are also involved in the transition
towards reproduction in other insects, we compared our results with
those from studies conducted in \emph{Anopheles }and \emph{Drosophila}}.
The comparison of our results with those of a study on the effects
of mating in female \emph{Anopheles gambiae }mosquitoes for 1,682
orthologs ant-\emph{Anopheles} orthologs \citep{Rogers2008AnophelesMating}
revealed that genes whose level of expression increased after orphaning
in \emph{S. invicta} queens are enriched for genes that are upregulated
after mating in \emph{Anopheles} (36 observed, 20.6 expected, exact
one-sided Fisher test $p=8*10^{-5}$, genes listed in Supporting Table
5). \textcolor{blue}{There was no significant overlap between other
pairs of lists of genes from the two studies.} Six of the thirty-six
genes identified in both studies are part of the manually assigned
gene category\emph{ proteasome }(0.5 expected, exact one-sided Fisher
test $p=3*10^{-5}$). 

Similar gene expression studies were also performed in the fruitfly
\emph{Drosophila melanogaster}. We found no significant overlap between
expression changes due to orphaning in fire ant queens and changes
due to mating in female \emph{Drosophila} \citep{LawniczakBegun2004FlyMating,McGraw2004MatingSpermACPsFlies},
nor between orphaned fire ant queens and specific aspects of \emph{Drosophila}
mating: the mating process itself (without receiving sperm), receiving
sperm, or receiving particular accessory proteins normally part of
sperm \citep{McGraw2004MatingSpermACPsFlies}. 


\section{Discussion}

We used microarrays to conduct a genome-wide survey of gene expression
in virgin \emph{Solenopsis invicta }fire ant queens over the 24 hours
that follow orphaning from their mother queen. We identified five
genes that are consistently differentially expressed within six hours
of orphaning. These early response genes may be responsible for some
of the additional 292 gene expression changes that take place within
24 hours of orphaning. The annotations of the differentially expressed
genes indicate that they potentially are involved in many different
functions, including signaling reproductive status, reproductive development,
proteasomal activity, \textcolor{blue}{protein transport, and regulation
of chromatin structure and transcription.} We discuss each in turn.


\subsection{Genes potentially involved in signaling of reproductive status}

The pheromones that the mother queen uses to signal her presence and
fertility are currently unknown. Our study revealed that\emph{ Glutathione
S-transferase} (GST)\emph{ }is the only gene downregulated in virgin
queens 6 hours after orphaning. Furthermore, an additional GST\emph{
}as well as five \emph{Cytochrome P450}s are significantly downregulated
in virgin queens within 24 hours of orphaning. Both GSTs and \emph{Cytochrome
P450}s are known to be involved in degrading foreign and endogenous
compounds \citep{CytochromeP450Montellano2005}. \textcolor{blue}{We
speculate that the virgin queens may use these genes to degrade fertility
signals produced by the mother queen.} This could be important if
\textcolor{blue}{maternal fertility signals also triggered reproductive
development in the virgins}. Alternatively, virgin queens may produce
their own fertility signals, and simultaneously degrade them using
the GSTs and \emph{Cytochrome P450s,}\textcolor{blue}{{} hence permitting
them to avoid aggression from the workers yet be able} to rapidly
increase levels of fertility signals when orphaned.

We also identified three upregulated genes putatively related to olfactory
signals, two \emph{chemo-sensory proteins} (CSPs) and one \emph{odorant
binding protein} (OBP). \textcolor{blue}{The} CSPs and OBP may play
the roles of carrier proteins \citep{GotzekReview2007,OBPReviewPelosi2005}
possibly involved in the production of reproductive status signals.
Interestingly, the gene with the highest sequence similarity to the
OBP is \emph{Gp-9,} a gene \textcolor{blue}{that is } linked to odor
differences between queens \citep{KellerGreenBeard1998,GotzekReview2007}
and \textcolor{blue}{to }the selective execution of queens which lack
the small \emph{b} allele at this locus in multiple-queen colonies
of \emph{S. invicta }\citep{Ross:1998pi,KellerGreenBeard1998,ross:2002,kriegerRoss02,GotzekRoss2009}\emph{.
}The upregulated OBP could similarly be involved in the production
of a qualitative signal by virgin queens.


\subsection{Genes known to be involved in reproductive development in \textcolor{blue}{social
insects}}

The level of Juvenile Hormone (JH) increases with the onset of reproduction
in many female insects. \textcolor{blue}{In particular, high JH titers
have been linked to reproductive dominance in bumble bees as well
as in }\textcolor{blue}{\emph{Polistes }}\textcolor{blue}{wasps, but
not in honey bees where JH has been shown to regulate the labor tasks
between workers \citep[reviewed in][]{Robinson1997Juvenile-hormon}.
}After orphaning \textcolor{blue}{young }\textcolor{blue}{\emph{S.
invicta }}\textcolor{blue}{queens}, JH synthesis rate increases and
JH body content peaks prior to wing shedding (\citealp{brentVargoJIP2003}\textcolor{blue}{;
see also Figure \ref{fig:Timing-of-post-orphaning})}. The ectopic
application of synthetic JH to virgin queens leads to wing shedding
even if the mother queen is present \citep{VargoLaurel1994}, whereas
applying an inhibitor of JH synthesis represses wing shedding in orphaned
virgin queens \citep{Burns2002Identification-}. The fact that JH
level increases after orphaning is consistent with our findings that
\textcolor{blue}{four} genes putatively involved in JH degradation
are downregulated after orphaning. Indeed, downregulation of these
genes should lead to reduced JH degradation and thus to increased
JH levels. Our data also imply that JH degradation genes are highly
expressed before orphaning, and thus that JH is already being produced
and simultaneously degraded before orphaning. Thus, maintenance of
low JH levels in virgin queens prior to orphaning may be due to the
simultaneous production and degradation of JH. \textcolor{blue}{This
has also been suggested to occur in bumble bee workers by \citet{Bloch2000JH-JIP}
who found that the rate of }\textcolor{blue}{\emph{in vitro}}\textcolor{blue}{{}
JH synthesis does not reliably indicate hemolymph JH titers. Such
dual control of JH titer by simultaneous production and degradation
of JH is known to exist from studies in solitary insects \citep{Kort1981JHReview,Tobe1985CorpusAllatum}.}

Beyond the role of JH, two small-scale studies identified genes associated
with reproductive differences in ants. In \emph{S. invicta }queens,
participation in a mating flight triggers wing shedding and reproductive
development \citep{Tschinkel:2006book} and leads to the upregulation
\textcolor{blue}{of at least seven genes }\citep{TianVinsonCoates7genes.2004}.
One of these genes, \emph{Striated Muscle Activator of Rho Signaling}
(STARS), was also significantly upregulated in our study 6 and 24
hours after orphaning. Five of the remaining genes, \emph{Vitellogenin-1},\emph{
Vitellogenin-2},\emph{ Yellow-1},\emph{ Yellow-2} and \emph{Abaecin}
\textcolor{blue}{were more highly expressed after orphaning, although
not significantly so.} \textcolor{blue}{A study in the black garden
ant}\textcolor{blue}{\emph{ Lasius niger}}\textcolor{blue}{{} identified
seven genes more highly expressed in mature queens than in workers
\citep{jemielityCasteDifferences}. While none of these genes showed
significant expression differences in our study, the mean expression
level for four of them was non-significantly higher after orphaning
in }\textcolor{blue}{\emph{S. invicta}}\textcolor{blue}{. The remaining
three genes were respectively absent from the}\textcolor{blue}{\emph{
S. invicta }}\textcolor{blue}{microarray, similarly expressed, or
had non-significantly lower mean expression levels after orphaning. }


\subsection{Genes that are putatively proteasomal}

Genes with similarity to proteasomal genes were highly overrepresented
among the genes upregulated after orphaning. Proteasomes are responsible
for degrading unneeded proteins. The proteasomal genes could be involved
in degrading wing muscle tissue or storage proteins such as hexamerins
and vitellogenins that would liberate amino-acids that can be used
for reproductive development. Alternatively, the increased proteasomal
activity after orphaning may trigger changes in gene expression or
cellular proliferation via the respective degradation of transcriptional
repressors or specific cyclins. Both possibilities are coherent with
the overrepresentation of proteasomal genes among the genes that we
identified as being upregulated after orphaning in ant queens and
also after mating in bees and mosquitoes. This indicates that \textcolor{blue}{the
role }of proteasomal genes during the onset of reproductive development
may be evolutionarily conserved. Furthermore, we detected significant
downregulation of a gene with similarity to \emph{Cellular Repressor
of E1A-stimulated Genes 1} (\emph{CREG1}) after orphaning. \emph{CREG1}
has been shown to inhibit growth in human cancer cells and to inhibit
apoptosis of human muscle cells \citep{Creg2006}. The downregulation
and degradation of this gene in virgin fire ant queens may similarly
induce proliferation of ovarian tissue or the apoptosis of wing muscle
cells.


\subsection{Genes putatively involved in protein transport}

Genes sharing sequence identity with those involved in protein transport
were highly overrepresented among the genes upregulated after orphaning.
Proteins need to be shuttled between intracellular compartments for
post-translational modifications as well as signal transduction. Protein
transportation is also essential for communication between cells via
the secretion and uptake of proteins \citep{CellBiologyBook2000}.
The upregulation of putative protein transport genes in orphaned fire
ant queens could be involved in changes in neuronal activity \citep{Buckley2000RegulationNeuronTrafficking}
as a response to orphaning. Alternatively, they may be involved in
ovarian development. 


\subsection{Genes putatively involved in transcriptional changes and chromatin
remodeling}

Three lines of evidence indicate that major transcriptomic and epigenetic
changes are taking place after orphaning in virgin fire ant queens.
First, the upregulated genes include two putative \emph{RNA polymerase}
subunits as well as\textcolor{blue}{{} a putative }\textcolor{blue}{\emph{Mediator
complex subunit}}\textcolor{blue}{{} involved in protein-coding gene
transcription }\citep{mediator2005}. Second, a Zinc finger transcription
factor domain containing gene is downregulated, while \emph{STARS}
and a RING finger transcription factor domain containing gene are
upregulated. \emph{STARS} may induce wing muscle degradation as previously
suggested \citep{TianVinsonCoates7genes.2004}.\textcolor{blue}{{} Finally,}
genes similar to \emph{Chromobox Homolog protein 1 }and \emph{Nucleoplasmin-like
protein} are upregulated after orphaning. Both are important for chromatin
remodeling \citep{heteroChromatin2006,nucleoplasmins2007}. Some or
all of these gene expression changes could be related to the post-orphaning
increases in ovarian development and egg production \citep{VargoLaurel1994,Vargo1999PhysiolEntomol}. 




\section{Conclusion}

This study represents the first genome-wide survey of gene expression
changes in subordinate animals \textcolor{blue}{immediately following}
the sudden loss of the dominant \textcolor{blue}{individual}. We identified
297 genes differentially expressed within 24 hours of orphaning in
virgin \emph{S. invicta }queens. Many of the observed gene expression
changes are consistent with previous knowledge about the physiological
changes in virgin queens after orphaning, and some genes related to
the onset of reproductive development appear to be conserved across
species from ants to bees and even mosquitoes. Additionally, we detected
several genes possibly required for the perception or production of
olfactory signals. These genes may play roles in triggering the onset
of reproductive development in virgin queens or in signaling reproductive
status to nestmates. Finally, we found evidence for activation of
genes putatively involved in muscle degradation and ovarian development.
However, much work remains to truly understand the molecular-genetic
cascades of events involved in the competition for reproductive dominance
between virgin queens. It will be particularly fascinating to understand
the evolutionary pressures acting upon different genes involved in
this process. A further challenge will be identifying the basis by
which workers make decisions regarding which competing queens to execute
and which to keep.

\bibliographystyle{molecularEcology}
\bibliography{N2}



\section{Acknowledgments}

We would like to thank Kenneth G Ross and Dietrich Gotzek for help
collecting ants, Micah Gardner for help feeding the ants, Christine
LaMendola and the Lausanne DNA Array Facility for support with molecular
work, Darlene Goldstein, Frédéric Schütz and the Bioconductor mailing
list for statistical advice, Rob L Hammond, Julien Roux and the Keller,
Chapuisat and Robinson-Rechavi labs for stimulating discussions, and
Kenneth G Ross and D. DeWayne Shoemaker for their helpful comments
on earlier versions of this manuscript. This work was funded by grants
from the Swiss National Science Foundation and the Rectorate of the
University of Lausanne.


\section{Figure Legends}
\begin{description}
\item [{\textcolor{blue}{Figure~\ref{fig:Timing-of-post-orphaning}}}] \textcolor{blue}{Timeline
of post-orphaning events in the fire ant (based on the following studies:
\citet{Barker1979MuscleHistolysis,Blum1981Pheromonal-Cont,Blum1983Regulation-of-Q,VargoLaurel1994,Vargo1999PhysiolEntomol,Burns:2007fk}).}
\item [{Figure~\ref{Fig:numbersOfGenes}:}] Numbers of genes significantly
differentially expressed in young fire ant queens within six hours
(left) and 24 hours of orphaning (right).
\item [{Figure~\ref{fig:jhPlot}:}] Expression levels of genes related
to Juvenile Hormone (JH) metabolism and response in virgin fire ant
queens that are either still in presence of their mother queen or
have been orphaned for 6 or 24 hours. Only genes with multiple clones
on the microarray are shown. Error bars represent the standard error
of the mean expression levels as obtained by independent clones. Genes
for which at least one representative clone is significantly differentially
expressed after orphaning are indicated by triangles.
\end{description}

\section{Table Legend}
\begin{description}
\item [{Table~\ref{Table 1}:}] Gene Ontology annotations that are significantly
enriched among genes that are significantly upregulated or downregulated
after orphaning.
\end{description}
\end{linenumbers}


\section{Tables and Figures}

%
\begin{figure}[H]
\caption{\label{fig:Timing-of-post-orphaning}}


\includegraphics{\string"/Users/yannickwurm/Dropbox/DealationProjAnalysis/2009 - Paper/timecoursePaperResubmission/images/timelineHorizontal169mm\string".eps}
\end{figure}


%
\begin{figure}[H]
\centering{}\caption{}
\includegraphics{images/replaceVennDiagram-Samesize-crop}\label{Fig:numbersOfGenes}
\end{figure}


%
\begin{figure}[H]
\centering{}\caption{\label{fig:jhPlot}}
\includegraphics{images/selectedGenes\lyxdot JHseparated}
\end{figure}


%
\begin{table}[H]
\centering{}\caption{\label{Table 1} }
\includegraphics[width=1\linewidth]{images/BlastGOTable}
\end{table}



\section{Author Information Box}

This work is part of Y. Wurm's PhD thesis under the supervision of
L. Keller. Y. Wurm and J. Wang use genetic tools to study the social
lives of ants. L. Keller works on various aspects of evolutionary
ecology such as reproductive skew, sex allocation, caste determination
as well as the molecular basis of aging and behavior in ants.


\section{Supporting Information}
\begin{description}
\item [{Supporting~Figure~1:}] Graphical representation of microarray
hybridizations. \textcolor{blue}{The unit of biological replication
is the colony; each pool of five queens was hybridized to two different
microarrays.} Each vertice represents an amplified RNA sample and
each edge represents a microarray hybridization (a total of $3*8=24$
hybridizations were conducted). Cy3-labeled samples are at the tails
and Cy5-labeled samples are at the heads of arrows.
\item [{Supporting~Table~1:}] List \textcolor{blue}{and annotations
}of all fire ant genes significantly upregulated for at least one
of the following comparisons: 6h vs 0h, 24h vs 6h, 24h vs 0h. Some
genes are significant according to multiple microarray clones.
\item [{Supporting~Table~2:}] List \textcolor{blue}{and annotations}
of all fire ant genes significantly downregulated for at least one
of the following comparisons: 6h vs 0h, 24h vs 6h, 24h vs 0h. Some
genes are significant according to multiple microarray clones.
\item [{Supporting~Table~3:}] List \textcolor{blue}{and annotations}
of all fire ant genes significantly upregulated for at least one of
the following comparisons: 6h vs 0h, 24h vs 6h, 24h vs 0h and also
significantly higher in brains of honey bee queens than reproductive
workers
\item [{Supporting~Table~4:}] List \textcolor{blue}{and annotations}
of all fire ant genes significantly upregulated for at least one of
the following comparisons: 6h vs 0h, 24h vs 6h, 24h vs 0h and also
significantly upregulated after mating in honey bee queens
\item [{Supporting~Table~5:}] List \textcolor{blue}{and annotations}
of all fire ant genes significantly upregulated for at least one of
the following comparisons: 6h vs 0h, 24h vs 6h, 24h vs 0h and also
significantly upregulated in\emph{ Anopheles gambiae} females in response
to mating according to Vectorbase gene expression data
\item [{Microarray~Data:}] Will be uploaded to the Gene Expression Omnibus
database (access information will be put here).
\end{description}

\end{document}
